\chapter*{Úvod}
\addcontentsline{toc}{chapter}{Úvod}

<Popis IoT>
TODO

<Popis embedded systémů>
Embedded systémy jsou v dnešní době velice rozšířené.

Firmware je sice přeložen pro STM3210C-Eval desku, nicméně s minimální změnou by
se dalo použít i jiné zařízení od STM.

Výhoda použití STM3210C-Eval spočívá zejména v tom, že kromě knihoven pro ovládání
všech periferií poskytuje výrobce také BSP (board support package) obsahující
knihovny pro ovládání všech komponent obsažených na desce jako je například LCD
displej.
Kdybychom použili pouze procesor od STM a zbytek desky si navrhli sami, pravděpodobně
bychom ušetřili v oblasti hardware, ale museli bychom naprogramovat vlastní BSP.

\section{Cíle práce}
Cílem práce je vytvořit firmware pro embedded zařízení chovající se jako termostat
a software pro server umožňující správu tohoto zařízení na dálku.

STM3210C-EVAL je vývojová deska od výrobce STM. Má na sobě mikroprocesor STM32F107VC,
který patří do rodiny [mikro]procesorů STM32F1.

Zařízení je osazeno mnoha periferiemi, ze kterých pro naše účely využijeme Ethernet,
dotykový LCD displej, EEPROM, RTC a GPIO.

LCD displej je k procesoru napojen pomocí SPI (Serial peripheral interface).
Pro naše účely nebudeme využívat dotykovou obrazovku na displeji, a to kvůli nepřesnosti a
zvýšení složitosti softwarového řešení.
Zařízení stačí ovládat pomocí joysticku umístěného rovnou pod displejem.

Pomocí GPIO (general purpose input/output) pinů jsou k desce připojeny teplotní sensor
Dallas(?) DS18B20+ a relé modul.
Na každém GPIO pinu lze nastavit směr vysílání, frekvenci (25 MHz, 50 MHz) a ???

\section{Struktura práce}
