\chapter*{Úvod}
\addcontentsline{toc}{chapter}{Úvod}

Termín \emph{embedded} zařízení označuje zařízení, které má typicky relativně omezené výpočetní schopnosti,
zejména ve smyslu výkonnosti procesoru a velikosti paměti.
Funkcionalita, kterou mohou embedded zařízení mít sahá od té nejjednodušší jako je třeba mikrokontrolér
v pračce, až po komplexní, jako jsou třeba chytré hodinky.
S tím přímo souvisí i architektura jejich software.
Na těch nejjednodušších, do nekonečna vykonávajících jedinou úlohu, běží v cyklu jediný software
nazývaný v takovém případě \emph{firmware}.
Na komplexnějších, typicky hardwarově výkonnějších zařízeních už může běžet nějaký vyšší operační systém
typu Linux nebo Windows, a tedy od klasických desktopových počítačů se příliš neliší.

Další důležitý termín pro tuto práci je \emph{IoT} (Internet of Things), což je označení pro síť,
která tyto embedded zařízení propojuje.

Tato práce se zabývá vývojem firmware pro embedded zařízení spadající do kategorie IoT s tím,
že toto zařízení komunikuje s centrálním serverem.


\section{Cíle práce}
Cílem práce je vytvořit firmware pro embedded zařízení chovající se jako termostat a umožňující
uživateli nastavit teplotu po libovolné časové intervaly přes den a zobrazující aktuálně naměřenou
teplotu, a software pro webový server umožňující správu tohoto zařízení na dálku.
Kromě firmware pro embedded zařízení a software pro webový server specifikuje tato práce také
komunikaci mezi těmito entitami a jejich synchronizaci, při které musíme vzít v úvahu to, že uživatel
může nastavovat různé hodnoty na embedded zařízení a na webovém serveru ve stejnou dobu.

\section{Struktura práce}
První kapitola nazvaná \uv{Technologie} představuje všechny technologie, které jsou v rámci celé práce
použity.
Patří sem mimo jiné volba konkrétního embedded zařízení.

Dále následují dvě hlavní kapitoly analýza a architektura, kde každá z nich má tři podsekce: \uv{STM},
\uv{Webový server} a \uv{Komunikace STM se serverem}.
Podsekce \uv{STM} rozebírá embedded část práce, \uv{Webový server} se zabývá návrhem a implementací webového serveru
a \uv{Komunikace STM se serverem} se zabývá propojením STM a webového serveru.
Kapitola analýza nabízí několik možností řešení každé podsekce a vybírá z nich ty nejvhodnější,
architektura se zabývá konkrétní implementací pro každou podsekci.

Jako další je kapitola \uv{Práce se systémem}, která popisuje použití v reálném prostředí, po ní kapitola
\uv{Vyhodnocení}, ve které je popsáno jakým způsobem se testuje funkcionalita celého systému, a nakonec
je \uv{Závěr}.
