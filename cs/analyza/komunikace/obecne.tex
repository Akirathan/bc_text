
% Jakou funkcionalitu bude náš systém potřebovat?
Náš komunikační systém bude potřebovat následující funkcionalitu:
\begin{itemize}
    \item Periodicky posílat aktuálně naměřenou teplotu z STM na server.
    \item Synchronizovat čas mezi serverem a STM
    \item Synchronizovat nastavení intervalů na STM a na serveru.
          Pro tento bod potřebujeme mít funkční synchronizaci času.
\end{itemize}

% definice actual a config
Dále budeme výše zmíněné měřitelné hodnoty, které STM měří a periodicky je posílá
serveru, nazývat [actual] a hodnoty, které se dají nastavit jak na serveru, tak na
STM [config] hodnoty.
% předpoklad přidávání nových actual a config položek
Při návrhu komunikace předpokládáme, že budeme chtít přidávat další config a actual
položky, takže v následujícím textu jsou intervaly zaměnitelné s jakoukoli
config položkou, totéž platí pro naměřenou teplotu a actual položky.

% STM a dynamická IP adresa
Předpokládáme, že webový server běží na statické, veřejné IP adrese.
Pokud STM připojíme ethernetovým kabelem do internetové sítě, bude mu přidělena dynamická IP adresa.
Nemá smysl aby uživatel, kromě toho že koupí STM, pořizoval i statickou IP adresu.
Vzhledem k tomu, že IP adresa STM je dynamická, server ji nemůže dopředu znát.
Což znamená, že iniciátor komunikace musí být STM.

% Volba HTTP vs custom protocol
Pro komunikaci mezi webovým serverem a STM můžeme vybrat buď HTTP nebo vlastní protokol.
Implementace a otestování vlastního protokolu by bylo příliš pracné, využijeme tedy existujícího HTTP.
Navíc nám v takovém případě stačí implementovat HTTP pouze na straně STM.

% Device ID
\paragraph{Device ID}
Vzhledem k tomu, že chceme aby server podporoval připojení více STM najednou, potřebujeme
tato STM nějakým způsobem identifikovat.
Otázka je kdy a kde tento identifikátor vzít.
Máme dvě možnosti:
\begin{itemize}
  \item Identifikátor je zakódovaný do firmware STM.
  \item Identifikátor se dynamicky vytvoří na serveru při prvním přístupu konkrétního STM.
\end{itemize}
Po serveru požadujeme, aby autentizoval STM tj. aby ověřil, že příchozí zprávy chodí skutečně
od nějakého STM.
Pokud bychom vytvářeli \emph{device ID} dynamicky na serveru, nemohli bychom ověřit, že právě
příchozí zpráva přišla od STM.
V našem případě bude lepší \emph{device ID} zakódovat do firmware každého STM a s tím rovnou
tento identifikátor uložit do databáze serveru.
Na rozdíl od privátního klíče (viz ...) nehrozí při uložení identifikátoru přímo do firmware STM
žádné bezpečnostní riziko.

%%%%%%%%%%%%%%%%%%%%%%%%%%%%%%%%%%%%%%%%%%%%%%%%%%%%%%%%%%%%%%%%%%%%%%%%%%%%%%%%%%%%%%%%%%%%
%                              SYNCHRONIZACE                                               %
%%%%%%%%%%%%%%%%%%%%%%%%%%%%%%%%%%%%%%%%%%%%%%%%%%%%%%%%%%%%%%%%%%%%%%%%%%%%%%%%%%%%%%%%%%%%

% Synchronizace config položek
Config data může uživatel nastavit jak na serveru, tak přímo na STM.
Proto je potřeba zavést synchronizaci.

\subsection{Synchronizace}

% nastavovat se můžu naráz na serveru i na STM
Uživateli dovolíme nastavovat config data naráz na serveru i na STM, platit bude ale pouze
nastavení, které bylo uloženo později (myšleno vzhledem k reálnému času).
Jiná, uživatelsky pravděpodobně příjemnější, možnost by byla nedovolit uživateli současně
nastavovat stejná config data na serveru i na STM.
Tato možnost je ovšem zbytečně složitá na implementaci.

% Synchronizace času - možnosti a proč je potřeba?
Pro synchronizaci config dat je potřeba, aby server i STM měli nastavený společný čas.
Vzhledem k tomu, že implementace \emph{NTP} (Network time protocol) ale i jednoduššího a podobného
protokolu \emph{SNTP} (Simple Network Time Protocol) je na STM pracná, zvolíme jiný, jednodušší způsob synchronizace
času mezi STM a serverem, který bude vycházet z \emph{TP} (Time protocol).

\subsubsection{Synchronizace času}
% vyjdeme z TP a upravíme ho pro naše potřeby
V rámci synchronizace času postačí sekundové rozlišení.
Vyjdeme z \emph{TP}, který má sekundové rozlišení, a ještě ho zjednodušíme pro naše účely.
Podobně jako je tomu u \emph{TP}, budeme ze serveru posílat počet sekund, které
uběhly od nějakého pevně stanoveného data.
Na rozdíl od \emph{TP} to ale ještě zjednodušíme tak, že tento počet sekund bude posílat server
v rámci první HTTP odpovědi.
Další rozdíl od \emph{TP} bute ten, že naše počáteční datum nebude \texttt{1.1.1900}, ale \texttt{1.1.1970}.
% definice timestamp
\textbf{Timestampem} budeme dále uvažovat číslo reprezentující počet sekund od \texttt{1.1.1970 00:00},
které se vejde do 4-bytové proměnné bez znamenénka.
% TODO: footnote?
Poznámka: výše zmíněná definice je totožná s \emph{Unix timestamp} (viz ...), který je nějakým způsobem
podporován ve většině programovacích jazyků, což nám usnadní programování serveru.

% uživatel musí nastavit čas při zapnutí
Při zapnutí STM musí uživatel nastavit čas. (TODO: viz ...)
Nemá možnost nastavit datum - to je implicitně nastaveno na \texttt{1.1.1970} (dále označováno jako \texttt{REF\_DATE}).
Je to nutné kvůli tomu, aby STM vědělo který interval je zrovna nastavený.
Uživatel může nastavit libovolný čas, který se nijak neshoduje s realitou, jde jen o to,
aby STM vědělo, jakou teplotu má udržovat.
% correct_bit
Když uživatel nastaví config data (intervaly), tak k nim STM uloží aktuální timestamp
a ještě k tomu příznak, který značí, že timestamp je synchronizovaný.
Tento příznak budeme dále označovat jako \textbf{correct\_bit}.

% Problém zpožděného času po synchronizaci
V souvislosti se vzdáleností STM od serveru a se zpožděním síťové komunikace se může
stát, že čas na STM bude o pár sekund opožděný (nemělo by to ale být více než 2 sekundy -
viz \texttt{[packet\_over\_world\_benchmark]}).
Kdyby se uživateli například podařilo na serveru uložit nové nastavení intervalů
a pár mikro sekund poté nastavit nové intervaly na STM, mělo by STM správně nahrát
intervaly na server, namísto toho ale intervaly stáhne ze serveru a svoje nastavení přepíše.
Předpokládejme, že uživatel toto dělat nebude.

% Poznámky o time zone
Z definice timestamp vyplývá, že STM nerozlišuje jednotlivé časové zóny.
Pro jednoduchost budeme čas na STM interpretovat jako čas v zóně \emph{UTC+2}.
Pro správnou interpretaci času na STM v různých časových zónách bychom museli implementovat
sofistikovanější časový protokol typu \emph{NTP}.
