
\subsection{Zabezpečení}

% Co požadujeme
\paragraph{Požadavky}
\begin{itemize}
    \item Autentikace zpráv. Potřebujeme zajistit, aby server přijímal pouze zprávy od STM a nikoho jiného.
    \item Šifrování zpráv. Z hlediska zabezpečení by bylo nejlepší kdybychom šifrovali jak těla HTTP
      zpráv, tak jejich hlavičky.
      Pro jednoduchost implementace budeme šifrovat pouze těla zpráv.
\end{itemize}


% použití HTTPS?
\paragraph{HTTPS}
Komunikace webového serveru s STM by mohla být kompletně vedena v HTTPS.
Pro HTTPS potřebujeme na STM TLS.
To je poskytováno například knihovnou Cyclone-SSL \cite{Cyclone-SSL}, která je závislá na knihovně
Cyclone-TCP/IP \cite{Cyclone-TCP/IP}.
My už jsme ovšem pro STM použili knihovnu LwIP, protože STM poskytuje ethernetový ovladač přímo
pro tuto knihovnu a tím pádem můžeme LwIP rovnou použít.
Kdybychom ale chtěli použít Cyclone-TCP/IP, tak bychom museli ještě implementovat ovladač pro
tuto knihovnu.
Závěr je tedy takový, že HTTPS používat nebudeme.

% Symetrické VS asymetrické šifrování
\paragraph{Symetrické nebo asymetrické šifrování}
Kdybychom chtěli použít pouze \emph{asymetrické} šifrování, server by měl privátní klíč a každé STM
by šifrovalo komunikaci veřejným klíčem.
Problém tohoto řešení je nedostatečná autentizace - veřejný klíč může získat každý, nejen STM.
Pro naše účely je tedy lepší použít \emph{symetrické} šifrování.
Dodejme ještě že střídání asymetrického a symetrického šifrování, podobně jako je tomu u výše zmíněného
HTTPS, jsme zavrhli zejména kvůli složitosti.
% jaké šifrování? --> DES
Potřebujeme, aby uživatel mohl pohodlně zadat klíč do STM.
Bude se nám tedy hodit šifrování s krátkým klíčem, což je například DES, které má
klíč délky 8 bytů.
Pro naše využití to je dostačující zabezpečení.


% Kde vzít klíče?
Pro symetrické šifrování potřebujeme, aby existovalo tolik párů klíčů, jako je STM zařízení.
Jeden z páru má server a druhý STM.
Následující text se zabývá problémem kde a kdy vzít tyto klíče.
\\ \\ 
Máme dvě možnosti:
\begin{itemize}
    \item STM má zabudovaný klíč ve svém firmware a server má databází všech těchto klíčů.
    \item Server generuje klíč na žádost uživatele.
\end{itemize}

% Proč je hardcoded klíč špatný nápad?
Mít klíč zabudovaný do firmware STM není příliš vhodné pro případy, kdy by si uživatel pořídil STM
\uv{z druhé ruky}.
Zejména je zde nižší zabezpečení nového vlastníka STM, bývalý vlastník by totiž teoreticky
mohl z firmware dostat podobu klíče a podvrhovat komunikaci novému vlastníkovi.

Proto zvolíme řešení s generováním nových klíčů na žádost uživatele.

% Generování klíčů
\subsubsection{Generování klíčů}

% Jak server může přiřadit klíče jednotlivým STM?
% predem, resp. až po přijetí connect
Server si ke každému STM potřebuje přiřadit klíč, což může udělat nejdřív v momentě, kdy
mu od daného STM přijde \emph{connect} zpráva (což je první zpráva, kterou STM pošle - viz
\ref{kap:specifikace-http-zprav}).
I kdyby dopředu věděl, který klíč patří kterému STM, nemůže vědět od kterého STM mu přišla
\emph{connect} zpráva protože dopředu nezná IP adresy jednotlivých STM.

% Postup generování klíčů
\subparagraph{Postup generování klíčů}
\begin{enumerate}
    \item Server vygeneruje klíč s omezenou životností na žádost uživatele.
    \item Uživatel zadá klíč do STM.
    \item STM pošle šifrovanou \emph{connect} zprávu serveru.
    \item Server se pokouší dešifrovat zprávu všemi dosud nespárovanými klíči, dokud nedostane korektní
        \emph{connect} zprávu tj. korektní HTTP hlavičku s tělem obsahujícím existující \emph{device ID}.
    \item Server má sprárované \emph{device ID} s klíčem a IP adresou.
\end{enumerate}

Výše zmíněný mechanismus generování klíčů a obecně vzato symetrické šifrování v tomto pojetí sice
přidává bezpečnost do komunikace, ovšem na úkor rychlosti připojení STM k serveru.
Server totiž dostane po každé \emph{connect} zprávu z \uv{neznámé} IP adresy a musí se tudíž
pokusit tuto zprávu dešifrovat postupně jednak všemi dosud nespárovanými klíči, ale také klíči
ze všech offline STM.
Proces připojení STM k serveru může být tedy poměrně zdlouhavý, obzvlášť v případě kdy je mnoho
STM offline.
To ale v našem případě nevadí.
