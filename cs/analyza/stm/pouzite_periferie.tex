
\subsection{Použité periferie}
V této části textu vypíšeme periferie na STM (tedy na desce STM3210C-Eval board), které jsou
pro naši aplikaci nejdůležitější a stručně popíšeme její fungování.
Nemá smysl zde opisovat technické detaily z referenčního manuálu.

% EEPROM
\subsubsection{EEPROM}
EEPROM je 64 KB persistentní uložiště dat na I2C sběrnici.
Uložit do něj můžeme de-facto cokoli a kdykoli v libovolném formátu.
V našem případě do této paměti budeme ukládat nastavení teplotních intervalů a privátní klíč.

% Časovače
\subsubsection{Časovače}
Na STM je několik časovačů různých typů lišících se podle toho co všechno podporují.
Velice zjednodušeně se na časovač dá dívat jako na komponentu, která má tři základní
16-bitové registry: prescaler, counter a auto-reload.
Prescaler dělí frekvenci časovače tj. jak rychle časovač zvyšuje counter.
Counter registr se zvyšuje nebo snižuje o jedna každý tik časovače.
Perioda časovače vyprší tehdy, když counter registr přeteče resp. podteče hodnotu
z auto-reload registru - podle toho, jestli je časovač nastavený aby counter zvyšoval nebo snižoval.
Časovač se dá nastavit tak, aby při vypršení své periody vygeneroval \emph{Update event} přerušení,
které se nám bude hodit při spouštění různých tasků (tasky deklarujeme jako handlery tohoto přerušení).

% RTC
\subsubsection{RTC}
% Obecný popis RTC
RTC (real-time clock) je nezávislý časovač.
Od ostatních časovačů se odlišuje tím, že se na něm dá snadno nastavit perioda
s délkou jedné sekundy a že je schopen provozu i pokud je napájen pouze z
baterie.

Jako zdroj hodinového signálu může mít HSE, LSE nebo LSI oscilátory.
V našem případě je nejvhodnější zvolit jako zdroj hodinového signálu LSI oscilátor,
protože díky němu můžeme snadno nastavit RTC tak, aby svůj čítač zvyšovalo
každou sekundu.

RTC má 32-bitový čítač, do kterého můžeme zapsat libovolný čas a později z
něj číst aktuální čas.

% Second interrupt
\paragraph{hardwarová přerušení}
RTC dále lze nastavit tak, aby při každém zvýšení čítače generovalo hardwarové
přerušení (second interrupt).
V naší aplikaci je sekundové přerušení žádoucí.

% Alarm, probuzení ze stand-by režimu
Další důležitá vlastnost RTC je, že je schopné probudit desku ze stand-by režimu
RTC je tedy schopné provozu i pokud je napájené pouze z baterie, respektive
zvyšuje pouze čítač a porovnává s přednastavenou hodnotou alarmu, a pokud je
čítač vyšší nebo stejný, probudí desku.
Což znamená, že bychom mohli desku pravidelně přepínat do režimu stand-by, ve
kterém je spotřeba energie minimální.
Náš systém ale očekává stálé napájení, zejména kvůli ethernetové periferii,
tuto vlastnost RTC tedy nevyužíváme.

% jak budeme RTC používat
\paragraph{Použití}
V naší aplikaci používáme pro synchronizaci času se serverem \emph{Unix timestamp}.
Vzhledem k tomu, že do RTC čítače můžeme kdykoli zapsat, můžeme do něj zapsat tento timestamp, který
získáme ze serveru a RTC bude čítač každou sekundu zvyšovat.
Tímto jednoduchým způsobem umožníme synchronizaci času na serveru a na STM.

% GPIO
\subsubsection{GPIO}
Na STM desce je celkem 100 GPIO pinů.
Každý z těchto pinů se dá nastavit jako vstupní nebo výstupní, navíc má k sobě interně připojený
rezistor.
Pro teplotní senzor budeme využít 3 GPIO piny - napájení, uzemnění a datový pin.
Komunikace s teplotním senzorem je specifikována \emph{one wire} protokolem a vyžaduje, aby na
datovém pinu byl připojený pull-up rezistor.
Tento datový pin stačí přepínat z režimu "input pull-up" tj. vstupní režim s připojeným pull-up
rezistorem, do režimu "output push-pull" tj. výstupní režim bez připojeného rezistoru.

% Ethernet
\subsubsection{Ethernet}
Vzhledem k veliké složitosti této periferie ji zde nebudeme popisovat a pouze řekneme, že ji
budeme používat.
