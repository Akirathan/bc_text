\subsubsection{Software}
Tato kapitola se zabývá tím, jak může vypadat výsledný software resp. firmware na STM.
Součástí toho je krátký přehled knihoven, které můžeme k vývoji využít a jejich porovnání.

% RTOS - použít či nepoužít
\paragraph{RTOS}
Existuje více variant RTOSu.
Pro naše účely bychom mohli použít například FreeRTOS (TODO: reference).
FreeRTOS funguje v podstatě pouze jako plánovač úloh - v naší aplikaci definujeme entry point
úlohy, její prioritu apod. a pak ji spustíme.
% RTOS a LwIP (jeho výhody)
Největší výhoda FreeRTOSu je při použití TCP/IP komunikace, resp. LwIP knihovny.
Umožňuje nám totiž programovat v tzv. LwIP sekvenčním API tj. vytvoření nového TCP spojení
a následném poslání dat můžeme naprogramovat v jednom bloku kódu.
LwIP při použití s RTOSem totiž pro příjem a odesílání paketů vytvoří nový task.
Na druhou stranu při použití LwIP bez RTOSu musíme programovat v tzv. callback API.
V tomto případě reaguje LwIP na události jako je například přijetí paketu, navázání TCP spojení, atd.,
tak, že zavolá uživatelem specifikovanou callback funkci.
Což znamená, že na zdánlivě velice jednoduchou úlohu typu: vytvoř TCP spojení a pošli 20 bajtů dat
potřebujeme alespoň tři odlišné funkce.
% Nevýhody RTOSu
Nevýhoda RTOSu nastává v momentě, kdy chceme aby víc různých tasků přistupovalo ke stejným datům
a musíme tasky synchronizovat.
To může být zdrojem špatně replikovatelných a laditelných chyb.
% RTOS je pro nás zbytečný
Použití RTOSu pro naši aplikaci je zbytečné.
Přestože bychom mohli několik "tasků", které v rámci naší aplikace vykonáváme, pohodlně vytvořit
a spustit s použitím RTOSu, bohatě nám postačí tyto "tasky" spouštět při vypršení hardwarových
timerů.



\paragraph{GUI}
Pro vývoj uživatelského rozhraní můžeme použít buď knihovnu typu STemWin (odkaz), nebo můžeme
vytvořit vlastní jednoduchou knihovnu.
Vzhledem k možných komplikacím při integraci tak veliké knihovny jako je STemWin a tomu, že
předpokládáme velice jednoduché GUI na STM, je lepší implementovat vlastní knihovnu.
