\subsection{Software}
Tato kapitola se zabývá tím z jakých komponent můžeme poskládat výsledný firmware na STM.
Součástí toho je krátký přehled knihoven, které můžeme použít k vývoji, a jejich porovnání.

% STM32CubeF1
\subsubsection{STM32CubeF1}
Výrobce STM poskytuje pro rodinu procesorů F1 balíček \emph{STM32CubeF1} obsahující různé knihovny
% TODO: footnote: rodinou procesorů F1 máme na mysli procesory typu STM32F1xx
pro abstrakci hardware na deskách obsahujících procesory F1.
Zde je důležité zmínit výhodu použití nezměněné desky STM3210C-Eval board od STM oproti poskládání
vlastní desky z různých hardware komponent.
Kromě knihovny \emph{HAL} (Hardware abstraction layer), která poskytuje abstrakci nad všemi
periferiemi na desce, obsahuje totiž \emph{STM32CubeF1} i knihovny \emph{BSP} (Board support package),
které poskytují abstrakce pro rozšíření specifická pro jednotlivé desky.
V našem případě je to například displej a joystick.

% RTOS - použít či nepoužít
\subsubsection{RTOS}
Existuje více variant RTOSu.
Pro naše účely bychom mohli použít například FreeRTOS (TODO: reference).
FreeRTOS funguje v podstatě pouze jako plánovač úloh - v naší aplikaci definujeme entry point
úlohy, její prioritu apod., a pak ji spustíme.
% RTOS a LwIP (jeho výhody)
Největší výhoda FreeRTOSu je při použití TCP/IP komunikace, resp. LwIP knihovny.
Umožňuje nám totiž programovat v tzv. LwIP sekvenčním API tj. vytvoření nového TCP spojení
a následném poslání dat můžeme naprogramovat v jednom bloku kódu.
LwIP při použití s RTOSem totiž pro příjem a odesílání paketů vytvoří nový task.
Na druhou stranu při použití LwIP bez RTOSu musíme programovat v tzv. callback API.
V tomto případě reaguje LwIP na události jako je například přijetí paketu, navázání TCP spojení, atd.,
tak, že zavolá uživatelem specifikovanou callback funkci.
Což znamená, že na zdánlivě velice jednoduchou úlohu typu: vytvoř TCP spojení a pošli 20 bajtů dat
potřebujeme alespoň tři odlišné funkce.
% Nevýhody RTOSu
Nevýhoda RTOSu nastává v momentě, kdy chceme aby víc různých tasků přistupovalo ke stejným datům
a musíme tasky synchronizovat.
To může být zdrojem špatně replikovatelných a špatně laditelných chyb.
% RTOS je pro nás zbytečný
Použití RTOSu pro naši aplikaci je zbytečné.
Přestože bychom mohli několik \uv{tasků}, které v rámci naší aplikace vykonáváme, pohodlně vytvořit
a spustit s použitím RTOSu, bohatě nám postačí tyto \uv{tasky} spouštět jako interrupt handlery některých
hardware timerů.

% GUI - knihovny pro GUI
\subsubsection{GUI}
Pro vývoj uživatelského rozhraní můžeme použít buď knihovnu typu STemWin (odkaz), nebo můžeme
vytvořit vlastní jednoduchou knihovnu.
Vzhledem k možných komplikacím při integraci tak veliké knihovny jako je STemWin a tomu, že
předpokládáme velice jednoduché GUI na STM, je lepší implementovat vlastní knihovnu.

\subsubsection{TCP/IP}
Požadavky na TCP/IP:
\begin{itemize}
  \item Pro jednoduchost předpokládejme, že server má statickou, veřejnou IPv4 adresu a nám tedy
    postačí podpora pro IPv4.
  \item TCP. To potřebujeme kvůli protokolu HTTP.
  \item Bylo by vhodné aby knihovna uměla i \emph{DNS}, protože zakódovat IP adresu serveru do firmware
    STM by bylo jednak velice nepraktické a jednak by to neumožňovalo přesunout server na jinou
    IP adresu.
  \item DHCP - určitě totiž nechceme kódovat IP adresu STM do jeho firmware, mohlo by totiž dojít
    ke kolizi s jiným STM nebo s úplně jiným zařízením například tiskárnou nebo osobním počítačem
    v rámci sítě, kde se STM vyskytuje.
\end{itemize}
% je víc knihoven, ale nemá smysl se jimi zabývat - vezmeme prostě LwIP.
Na výběr máme z většího množství knihoven určených primárně pro embedded zařízení.
Vzhledem k tomu, že součástí STM32CubeF1 je integrační vrstva pro knihovnu LwIP, která podporuje
všechny výše zmíněné požadavky a ještě spoustu dalších, nemá smysl se zabývat porovnáváním LwIP
s jinými TCP/IP knihovnami.
Navíc ani nepožadujeme aby TCP/IP knihovna byla příliš efektivní.
V našem případě je tedy nejlepší použít knihovnu LwIP, s tím, že její integrace nám zabere minimálně
času.

