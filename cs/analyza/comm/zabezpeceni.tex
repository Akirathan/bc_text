
\subsection{Zabezpečení}

% Co požadujeme
\paragraph{Požadavky}
\begin{itemize}
    \item Autentikace zpráv: potřebujeme zajistit, aby server přijímal pouze zprávy od STM a nikoho jiného.
    \item Šifrování zpráv: šifrovat budeme jak hlavičky, tak těla HTTP zpráv. Z hlediska bezpečnosti
        je to lepší než šifrovat pouze těla.
\end{itemize}


% použití HTTPS?
\paragraph{HTTPS}
Komunikace webového serveru s STM by mohla být kompletně vedena v HTTPS.
Pro HTTPS potřebujeme na STM TLS.
To je poskytováno například knihovnou Cyclone-SSL, která je závislá na knihovně Cyclone-TCP/IP.
My už jsme ovšem pro STM použili knihovnu LwIP, protože STM poskytuje ethernetový driver přímo
pro tuto knihovnu a tím pádem můžeme LwIP rovnou použít.
Kdybychom ale chtěli použít Cyclone-TCP/IP, tak bychom museli ještě implementovat driver pro
tuto knihovnu.
Závěr je tedy takový, že HTTPS používat nebudeme.

% Symetrické VS asymetrické šifrování
\paragraph{Symetrické nebo asymetrické šifrování}
Kdybychom chtěli použít pouze \emph{asymetrické} šifrování, server by měl privátní klíč a každé STM
by šifrovalo komunikaci veřejným klíčem.
Problém tohoto řešení je nedostatečná autentizace - veřejný klíč může získat každý, nejen STM.
Pro naše účely je tedy lepší použít \emph{symetrické} šifrování.
% jaké šifrování? --> DES
Potřebujeme, aby uživatel mohl pohodlně zadat klíč do STM.
Bude se nám tedy hodit šifrování s krátkým klíčem, což je například DES, které má
klíč délky 8 bitů.
Pro naše využití to je dostačující zabezpečení.


% Kde vzít klíče?
Pro symetrické šifrování potřebujeme, aby existovalo tolik párů klíčů, jako je STM zařízení.
Jeden z páru má server a druhý STM.
Následující text se zabývá problémem kde a kdy vzít tyto klíče.
\\ \\ 
Máme dvě možnosti:
\begin{itemize}
    \item STM má hardcoded klíč ve svém FW a server má databází všech těchto klíčů.
    \item Server generuje klíče na žádost uživatele.
\end{itemize}

% TODO: proč je hardcoded klíč špatný nápad?

% Generování klíčů
\subsubsection{Generování klíčů}

% Obecné vlastnosti generování klíčů
Nevýhoda generování klíčů na serveru je ta, že server nemůže vědět, kterému STM má klíč přiřadit.

% Jak server může přiřadit klíče jednotlivým STM?
% predem, resp. až po přijetí connect
Server si ke každému STM potřebuje přiřadit klíč.
I kdyby dopředu věděl, který klíč patří kterému STM, nemůže vědět od kterého STM mu přišla
\emph{connect} zpráva (což je první zpráva, kterou STM pošle - viz ...), protože dopředu nezná
IP adresy jednotlivých STM.
Což znamená, že není důležité, jestli má klíče k zařízením přiřazené předem, nebo jestli
si je přiřadí až po tom, co od jednotlivých zařízení dostane \emph{connect} zprávu.

% Postup generování klíčů
\subparagraph{Postup generování klíčů}
\begin{enumerate}
    \item Server vygeneruje klíč s omezenou životností na žádost uživatele.
    \item Uživatel zadá klíč a IP adresu serveru do STM.
    \item STM pošle šifrovanou \emph{connect} zprávu serveru.
    \item Server bude dešifrovat zprávu všemi dosud nespárovanými klíči, dokud nedostane korektní
        \emph{connect} zprávu tj. korektní HTTP hlavičku s tělem obsahujícím existující Device ID.
    \item Server má sprárované Device ID s klíčem a IP adresou.
\end{enumerate}



% UID (TODO: tohle bude v architektuře)
%Každé STM zařízení má v sobě uložený 96-bitový unikátní identifikátor.