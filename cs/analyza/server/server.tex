
% Funkcionalita
\subsection{Funkcionalita}
Požadavky na webový server jsou:
\begin{itemize}
  \item Podpora více STM pro jednoho uživatele.
  \item Připojení více STM najednou.
  \item Zobrazení stavu všech uživatelových STM.
  \item Možnost nastavit různou teplotu po různé časové intervaly.
  \item Autentikace STM.
  \item Autorizace uživatele.
\end{itemize}
% Obecné věci
Webový server slouží jako dálkové rozhraní mezi uživatelem a zařízením.
Uživatel může odkudkoli, kde má připojení k internetu, zkontrolovat činnost zařízení
a případně ho přenastavit.

Server podporuje pouze jeden druh zařízení (STM), nicméně je naprogramován
tak, že pro případné rozšíření na více druhů zařízení nemusíme v jeho softwaru dělat veliké změny.

% Udržování spojení s STM
\paragraph{Udržování spojení s STM}
Server potřebuje s STM udržovat spojení, resp. potřebuje vědět, jestli je STM k serveru připojeno,
nebo jestli je offline.
Bylo by také vhodné, aby server tento stav dostatečně často aktualizoval a mohl ho tím pádem s malou
odezvou zobrazit uživateli.
Kapitola \ref{analyza-komunikace} detailně specifikuje komunikaci mezi serverem a STM.
Pro teď nám vystačí informace, že veškerá komunikace bude probíhat pomocí HTTP zpráv.
Ta je důležitá pro to, abychom se mohli rozhodnout jakým způsobem na serveru rozhodnout, zda je dané
STM zařízení offline nebo je k němu připojené.

Možnosti jak udržovat spojení s STM jsou:
\begin{enumerate}
  \item Využít vlastnosti persistent connection u HTTP/1.1 \cite{HTTP-persistent-connection}.
    S využitím této vlastnosti by server mohl STM deklarovat jako offline v momentě, kdy by uzavřel
    TCP spojení.
  \item S ohledem na jednoduchost klienta (STM), udržet komunikaci co nejjednodušší tj.
    na HTTP/1.0 s co nejmenším počtem header options ze strany klienta.
    V takovém případě je na serveru nutnost zavést něco jako mechanismus časovače, který se obnoví po každé,
    když přijde libovolná zpráva od daného STM, a v momentě kdy vyprší, označí STM jako offline.
  \item STM a server budou udržovat separátní TCP spojení na úplně jiném portu, které
    by po dobu své životnosti značilo pouze to, že STM je připojeno k serveru.
    Tato možnost je příliš komplikovaná, zejména kvůli tomu, že každé STM by si se serverem muselo domluvit
    unikátní port přes který toto nové spojení budou udržovat.
\end{enumerate}

V rámci dodržení strategie co nejjednoduššího firmwaru STM je řešení číslo 2 nejvhodnější.

% Použité technologie
\subsection{Použité technologie}

\subsubsection{Backend}
% Backend framework a proč ho používat?
Přestože očekáváme, že konečná webová aplikace bude poměrně malá a to jak požadovanou funkcionalitou, tak svým
vzhledem, použití některého z webových backend frameworků nám může velice usnadnit práci, například v těchto
aspektech:
\begin{itemize}
    \item Zabudovaná podpora uživatelů - nám jen stačí přidat uživatelům odkazy na jejich
        zařízení.
    \item Zabudovaná podpora databázových systémů - většina backend frameworků poskytuje vrstvu abstrakce nad
        databázovými systémy. V praxi to znamená, že nikde v kódu nemusíme přímo vytvářet SQL dotazy,
        stačí používat funkční API pro databáze v programovacím jazyce, ve kterém je daný framework napsaný.
\end{itemize}

% Různé frameworky a jaký je mezi nimi rozdíl
Existuje mnoho různých backend frameworků.
Jsou to například NodeJS napsaný v Javascriptu, Ruby on Rails napsaný v Ruby, Django \cite{Django} napsaný v Pythonu a
Laravel napsaný v PHP.
Všechny tyto frameworky nám mohou usnadnit práci podobným způsobem, pro naši aplikaci se liší především
tím, jaký programovací jazyk používají.

% Výběr Djanga
Vzhledem k tomu, že pro naši aplikaci je pro výběr konkretního frameworku relevantní pouze to, jaký programovací
jazyk používá, a vzhledem k tomu, že autor má předchozí zkušenosti s Pythonem, vybereme Django.

\subsubsection{Frontend}
% Bootstrap
Vytváření vzhledu webu si usnadníme použitím knihovny Bootstrap \cite{Bootstrap}.
Ta nám umožňuje snadno vytvořit moderně vypadající responsive webovou aplikaci. 

% interaktivní device stránka
Pro větší přehlednost požadujeme, aby se uživateli všechny jeho STM zobrazily na jednu stránku,
která bude alespoň trochu interaktivní.
Bylo by proto vhodné použít nějaký frontend framework, například ReactJS nebo AngularJS.
Integrace některého z těchto frontend frameworků do Djanga je ale příliš komplikovaná a náš web
příliš malý pro to, aby se jejich použití vyplatilo.
Proto budeme frontend vyvíjet pouze s pomocí Bootstrap a jQuerry \cite{jQuery}.
