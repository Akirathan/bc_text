\section{Webový server}

\subsection{Funkcionalita}
Požadavky na webový server jsou:
\begin{itemize}
  \item Podpora více STM pro jednoho uživatele.
  \item Připojení více STM najednou.
  \item Zobrazení stavu všech uživatelových STM.
  \item Možnost nastavit různou teplotu po různé časové intervaly.
  \item Autentikace STM.
  \item Autorizace uživatele.
\end{itemize}
% Obecné věci
Webový server slouží jako dálkové rozhraní mezi uživatelem a zařízením.
Uživatel může odkudkoli kde má připojení k internetu zkontrolovat činnost zařízení
a případně ho přenastavit.

Server podporuje pouze jeden druh zařízení (STM smart heating), nicméně je naprogramován
tak, že pro případné rozšíření na více druhů zařízení nemusíme v programu dělat veliké
změny.

\subsection{Použité technologie}

\subsubsection{Backend}
% Backend framework a proč ho používat?
Přestože očekáváme, že konečná webová aplikace bude poměrně malá a to jak požadovanou funkcionalitou, tak svým
vzhledem, použití některého z webových backend frameworků nám může velice usnadnit práci, například v těchto
aspektech:
\begin{itemize}
    \item Zabudovaná podpora uživatelů - nám jen stačí přidat uživatelům odkazy na jejich
        zařízení.
    \item Template system nám umožňuje v HTML stránkách používat speciální
        tagy, díky kterým můžeme například do HTML stránky vložit hodnotu proměnné, nebo ... (TODO)
        Template system nakonec vygeneruje validní HTML stránku.
    \item Zabudovaná podpora databázových systémů - většina backend frameworků poskytuje vrstvu abstrakce nad
        databázovýmy(?) systémy. V praxi to znamená, že nikde v kódu nemusíme přímo vytvářet SQL dotazy,
        stačí používat funkční API pro databáze v programovacím jazyce, ve kterém je daný framework napsaný.
\end{itemize}

% Různé frameworky a jaký je mezi nimi rozdíl
Existuje mnoho různých backend frameworků.
Jsou to například NodeJS napsaný v Javascriptu, Ruby on Rails napsaný v Ruby, Django napsaný v Pythonu a
Laravel napsaný v PHP. % TODO: formátovat jednotlivé názvy, případně přidat reference
Všechny tyto frameworky nám mohou usnadnit práci podobným způsobem, pro naši aplikaci se liší především
tím jaký programovací jazyk používají.

% Výběr Djanga
Vzhledem k tomu, že pro naši aplikaci je pro výběr konkretního frameworku relevantní pouze to jaký programovací
jazyk používá, a vzhledem k tomu, že autor má předchozí zkušenosti s Pythonem, vybereme Django.

\subsubsection{Frontend}
% Bootstrap
Vytváření vzhledu webu si usnadníme použitím knihovny Bootstrap.
Ta nám umožňuje snadno vytvořit moderně vypadající responsive webovou aplikaci. % TODO: co znamená responsive?

% interaktivní device stránka
Pro větší přehlednost požadujeme, aby se uživateli všechny jeho STM zobrazily na jednu stránku,
která bude alespoň trochu interaktivní.
Bylo by proto vhodné použít nějaký frontend framework, například ReactJS nebo AngularJS.
Integrace některého z těchto frontend frameworků do Djanga je ale příliš komplikovaná a náš web
příliš malý pro to, aby se jejich použití vyplatilo.
Proto budeme frontend vyvíjet pouze s pomocí Bootstrap a jQuerry.
