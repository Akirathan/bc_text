
Jak víme z předchozích kapitol, jednotlivé obrazovky na STM odpovídají jednotlivým
\texttt{Frame}s v kódu.
V této kapitole se budeme zabývat za jakých okolností se spustí jaké obrazovky.

% Stavy vzhledem ke GUI.
Nejprve specifikujeme v jakých stavech se STM může nacházet.
Toto jsou pouze stavy, které nás zajímavý v kontextu GUI - tj. v kontextu toho co všechno chceme
uživateli zobrazit.
Rozhodně se nejedná o souhrn všech možný stavů.
\subsection{Stavy STM}
\begin{itemize}
  \item Ethernetová periferie je korektně inicializována a kabel je připojen. Zkráceně budeme
        značit \texttt{ETH-up}.
  \item Ethernetový kabel je buď odpojen nebo Ethernetová periferie se z nějakého důvodu
        neinicializovala. Zkráceně budeme značit jako \texttt{ETH-down}.
  \item STM je připojeno k serveru. Zkráceně značíme \texttt{CONNECTED}.
  \item STM je odpojeno od serveru. Zkráceně značíme \texttt{DISCONNECTED}.
  \item STM se připojuje k serveru, s tím že je nastavený (TODO) nějaký timeout. Zkráceně značíme
        \texttt{CONNECTING}.
  \item V EEPROM nejsou uložena žádná konfigurační data intervalů. Tento stav nastává pouze v případě,
        kdy uživatel zapne STM poprvé.
  \item V EEPROM jsou uložena konfigurační data intervalů.
  \item Uživatel už zadal klíč (viz ...) do STM. Tento klíč se ukládá do EEPROM, aby ho uživatel
        nemusel zadávat opakovaně.
  \item Uživatel ještě klíč nezadal. To znamená, že se ještě nepokoušel připojit k serveru.
\end{itemize}

% Invarianty - co chceme aby v rámci GUI platilo.
\subsection{Invarianty}
ETH-up a ETH-down se může v podstatě libovolně střídat v běhu aplikace a chceme o tom uživateli
dát vědět.
Pokud se za běhu aplikaceme dostaneme do stavu \texttt{ETH-up}, dáme uživateli možnost připojit
se k serveru.
Pokud jsme ve stavu \texttt{CONNECTED}, nemůžeme se odpojit. Jedině pokud resetujeme celé STM.
Pokud STM vůbec není připojeno k internetu a EEPROM je prázdná, dáme uživateli možnost nastavit
intervaly.

% Popis přepínání mezi obrazovkami
\subsection{Popis přepínání mezi framy}
[TODO: vložený frame diagram]

MainFrame mimo jiné zobrazuje stav \texttt{CONNECTED} nebo \texttt{DISCONNECTED} nebo \texttt{CONNECTING}
podle výše zminěného popisu.
% tlačítko connect
Pokud je stav \texttt{DISCONNECTED} a zároveň \texttt{ETH-up}, objeví se v MainFrame tlačítko connect,
pomocí kterého se uživatel dostane na obrazovku KeyFrame, kam může zadat klíč vygenerovaný serverem a
připojit se tak na server.

% Výhody řešení
\subsection{Výhody řešení}
- Uživatel nemusí dvakrát zadávat klíč - ten se poprvé prostě uloží do EEPROM.
- Pokud je STM připojeno k internetu, rovnou se připojí k serveru, popřípadě zobrazí KeyFrame pokud
  ještě nebyl zadaný klíč.
  - Tím že se STM rovnou připojí k serveru, ušetří uživateli zadávání času a čas rovnou synchronizuje
    se serverem.
