\section{Technologie}
Tato kapitola se zabývá technologiemi, které máme v rámci naší práce k dispozici.

% Koncové zařízení
\subsection{Koncové zařízení}
Koncovým zařízením máme na mysli samotný termostat, který měří teplotu u uživatele doma.
% high-level pohled: embedded / linux
Zařízení má poměrně jednoduchou úlohu - má pouze měřit teplotu, umožnit uživateli nastavit teplotu po
různé časové intervlz a to všechno periodicky posílat na server.
K vykonání takto jednoduché úlohy nepotřebujeme příliš paměti, ani výkonný procesor.
Bohatě nám vystačí zhruba 256 KB paměti pro kód a 64 KB RAM paměti.
Po zařízení také požadujeme, aby nebylo příliš drahé a tím pádem neobsahovalo zbytečně výkonný
hardware, který nevyužijeme.
To rovnou vyřazuje z výběru zařízení typu Raspberry Pi (TODO: odkaz?).
% S Linuxem by bylo snadnější programování
Tím, že ušetříme na hardware, si zase ztížíme práci na software.
Kdybychom totiž použili zařízení typu Raspberry Pi, mohli bychom do něj, díky relativně
výkonnému hardware, nahrát operační systém Linux a tím pádem bychom měli k dispozici nejen
abstrakci nad hardware, ale také v podstatě libovolnou technologii, kterou bychom mohli použít
při programování našeho software.
Jediné co by mohlo být problematické s použitím Linuxu je komunikace s externími hardwarovými (?)
komponentami.


% Požadované IO koncového zařízení
\subsubsection{Vstupně-výstupní požadavky}
Vzhledem k tomu, že chceme aby zařízení zobrazovalo aktuální údaje a navíc aby uživatel mohl
různé údaje na zařízení nastavovat, potřebujeme aby zařízení mělo alespoň malý displej.
Pro vstup od uživatele bychom mohli použít buď sadu tlačítek nebo joystick což se dá považovat
za téměř ekvivalentní řešení.
Dále by se dal použít dotykový displej, ten je ale pro naše účely zbytečně nákladný.

% Vybereme STM3210C-Eval
V našem případě použijeme zařízení STM3210C-Eval board se zabudovaným 3,2 palcovým displejem,
joystickem, GPIO konektory a ethernetovým konektorem.
Povaha našeho firmware umožňuje použít jiné zařízení z rodiny STM32 pouze s drobnými modifikacemi.

STM3210C-Eval board nepatří mezi nejlevnější zařízení, na druhou stranu ale obsahuje obrovské
množství periferií které nevyužijeme - například CAN, Motor control, SD kartu, apod.

% TODO: formátovat poznámku
\strong{Poznámka}: ve zbytku textu budeme koncové zařízení označovat pouze jako \emph{STM}.

% teplotní senzor
\subsubsection{Externí komponenty}
\paragraph{Teplotní senzor}
Dále potřebujeme dostatečně přesný a zároveň levný teplotní senzor.
K tomu účelu nám vystačí DS18B20, který můžeme k téměř jakémukoli zařízení připojit pomocí tří
GPIO konektorů - napájení, uzemnění a data.
Jedna z výhod teplotního senzoru DS18B20 je ta, že sám provádí převod z analogové naměřené
teploty na digitální data.
Při maximální přesnosti měření na čtyři desetinná místa stupňů celsia trvá tento převod zhruba
750 ms.

% relé modul
\paragraph{Relé modul}
Jako poslední potřebujeme relé modul, který nám umožní spouštět připojený kotel.


% Server
\subsection{Webový server}
% Požadavky
Na webový server máme následující požadavky:
\begin{itemize}
  \item Podpora více zařízení pro jednoho uživatele.
  \item Zobrazování naměřené teploty a nastavených intervalů každého zařízení.
  \item Možnost přenastavit intervaly.
\end{itemize}

% kde poběží - koncové zařízení/externě?
Požadavek na podporu zobrazování hodnot z více zařízení najednou téměř znemožňuje použití koncového
zařízení jako webového serveru.
A to jednak malou výkonností koncového zařízení, ale také přílišnou složitostí řešení - museli bychom
se totiž rozhodovat, které koncové zařízení jednoho uživatele použijeme jako webový server a nebo bychom
museli uživateli dodat úplně jiné zařízení, na kterém webový server poběží.

V našem případě bude lepší použít právě jeden centrální webový server, který poběží na hardwaru typického
serveru.
Mimo jiné můžeme na tomto centrálním webovém serveru s výhodou využít architektury LAMP (TODO: odkaz),
která nám výrazně usnadní vývoj webu.
