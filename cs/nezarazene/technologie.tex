\subsection{Technologie}
Tato kapitola se zabývá technologiemi, které máme v rámci naší práce k dispozici.

% Koncové zařízení
\subsubsection{Koncové zařízení}
Koncovým zařízením máme na mysli samotný termostat, který měří teplotu u uživatele doma.
% požadované IO koncového zařízení
Vzhledem k tomu, že chceme aby zařízení zobrazovalo aktuální údaje a navíc aby uživatel mohl
různé údaje na zařízení nastavovat, potřebujeme aby zařízení mělo alespoň malý displej.
Pro vstup od uživatele bychom mohli použít buď sadu tlačítek nebo joystick což se dá považovat
za téměř ekvivalentní řešení.
Dále by se dal použít dotykový displej, ten je ale pro naše účely zbytečně nákladný.

% teplotní senzor
Dále potřebujeme dostatečně přesný a zároveň levný teplotní senzor.
K tomu účelu nám vystačí DS18B20, který můžeme k téměř jakémukoli zařízení připojit pomocí tří
GPIO konektorů - napájení, uzemnění a data.
Jedna z výhod teplotního senzoru DS18B20 je ta, že sám provádí převod z analogové naměřené
teploty na digitální data.
Při maximální přesnosti měření na čtyři desetinná místa stupňů celsia trvá tento převod zhruba
750 ms.

% relé modul
Jako poslední potřebujeme relé modul, který nám umožní spouštět připojený kotel.


V našem případě používáme STM3210C-Eval board.
Pro nahrávání firmware do této desky používáme ST-Link přítomný na STM32F4-Discovery.
Povaha našeho firmware umožňuje použít jiné zařízení z rodiny STM32 pouze s drobnými modifikacemi.

STM3210C-Eval board nepatří mezi nejlevnější zařízení, na druhou stranu ale obsahuje obrovské
množství periferií které nevyužijeme - například CAN, Motor control, ...

% Web server


