
\section{Webový server}

% Čím se v této kapitole budeme zabývat
% TODO: ...

Webový server je rozdělený do dvou na sobě nezávislých komponent \texttt{user\_interface} a
\texttt{stm\_communication}.
user_interface zprostředkovává veškerou komunikaci mezi uživatelem (klientem) a jeho STM
zařízeními sestávající z nastavování intevalů a čtení aktuálně naměřené teploty.
stm_communication přijímá aktuálně naměřenou teplotu od STM zařízení a vyměňuje si s nimi intervaly.
Obě komponenty čtou a zapisují data do společné databáze.
Tím je nepřímo zajištěna komunikace mezi uživatelem a jeho zařízením skrz server.


% TODO: obrázek databázového schématu
Databázové schéma (TODO: ref) ukazuje základní objekty a vztahy mezi nimi.

Na server se může registrovat libovolný uživatel zadáním uživatelského jména, hesla a potvrzením
emailové adresy.
Každý uživatel si může přidat omezeně mnoho zařízení.

% ConnectionManager
Chceme aby uživatel byl notifikován o nejnovějších událostech, které se staly na jeho zařízeních.
Základem je rozlišovat stavy, kdy je STM připojeno k serveru a kdy je offline.
Dále chceme tento stav dostatečně často aktualizovat.
Pro tyto účely slouží v rámci serveru třída \texttt{ConnectionManager}, která po 10 sekundách
neaktivity deklaruje STM jako offline.
Neaktivitou máme na mysli pravidelné neposílání dat z STM.

% Použití frontend frameworku by bylo vhodné.

