\section{Komunikace STM a web serveru}

% Actual a config položky
STM pracuje se dvěma typy dat:
1) Aktuálně naměřená data - to je v našem případě pouze naměřená teplota. Tyto data budeme nazývat [actual] data.
Ze serveru se tato data nedají nastavit.
2) Nastavitelná data - to je v našem případě pouze nastavení intervalů. Dále tyto data budeme nazávat [config] data.
Ze serveru se tato data běžně nastavují.

% V budoucnu je možnost přidat nové instance položek
V našem případě pracuje STM pouze s jednou instancí z každého typu dat.
Mohlo by se ale stát, že do budoucna budeme chtít přidat některá actual data jako je
třeba vlhkost vzduchu, popřípadě další config data.

% actual položky musíme posílat periodicky
Pro uživatelův přehled musí STM posílat actual data periodicky na server.
Výpadek těchto dat považujeme za chybu.

% Synchronizace config položek
Config data může uživatel nastavit jak na serveru, tak přímo na STM.
Proto je potřeba zavést synchronizaci.
Uživateli dovolíme nastavovat config data naráz na serveru i na STM, platit bude ale pouze
nastavení, které bylo uloženo později (myšleno vzhledem k reálnému času).
Jiná, uživatelsky pravděpodobně příjemnější, možnost by byla nedovolit uživateli současně
nastavovat stejná config data na serveru i na STM.
Tato možnost je ovšem zbytečně složitá na implementaci.

% Synchronizace času
Vzhledem k tomu, že implementace NTP (Network time protocol) ale i jednoduššího a podobného
protokolu TP (Time protocol) je na STM pracná, zvolíme jiný, jednodušší způsob synchronizace
času mezi STM a serverem.
% Možnosti synchronizace času??
TODO: ...

% řešení s posíláním timestampů
Předtím než dojde k samotné výměně config dat, je potřeba vyměnit timestamp těchto dat.
Fungování je jednoduché:
- ze strany A do strany B pošleme timestamp
- strana B porovná právě přijatý timestamp se svým timestampem a na základě toho buď pošle
    straně A svoji položku, nebo si od strany A vyžádá její položku.
V našem případě je jedno, kdo bude timestamp porovnávat - zvolíme tedy STM.
% efektivita řešení
Toto řešení mimo jiné dovoluje po modifikaci config dat na STM nebo na serveru
tyto data poslat pouze jednou a zbytečně nevytěžovat síťovou komunikaci periodickým
posíláním.


STM musí poslat config položku pouze v případě kdy je tato položka na serveru starší.


% Volba HTTP vs custom protocol
Pro komunikaci mezi webovým serverem a STM (tedy koncovým zařízením) můžeme vybrat buď HTTP nebo vlastní protokol.
Implementace a otestování vlastního protokolu by bylo příliš pracné, využijeme tedy existujícího HTTP.
Navíc nám v takovém případě stačí implementovat pouze na straně STM.

% Jakou funkcionalitu bude náš systém potřebovat?
Náš komunikační systém bude potřebovat následující funkcionalitu:
\begin{itemize}
    \item Posílat aktuálně naměřenou teplotu z STM na server.
    \item Synchronizovat nastavení intervalů na STM a na serveru.
\end{itemize}



% Jak budeme využívat HTTP?


% UID
Každé STM zařízení má v sobě uložený 96-bitový unikátní identifikátor.

% Connection fáze
Nejprve je potřeba vyřešit registrace resp. připojení STM k serveru.
Můžeme poslat (nezašifrovaný) požadavek typu:
\begin{code}
    GET /connect/<dev_id> 
\end{code}
...