\chapter*{Závěr}
\addcontentsline{toc}{chapter}{Závěr}

Cíle práce se dají rozdělit do tří částí: embedded část práce, web-server část práce a nakonec
komunikace mezi webový serverem a embedded zařízením.
Následuje shrnutí jakým způsobem jsme dosáhli cíle v rámci jednotlivých částí.

\paragraph{Embedded část práce}
V rámci této části práce jsme měli za cíl vytvořit firmware umožňující uživateli nastavovat
různé teploty po různé časové úseky během dne a zároveň zobrazovat aktuálně naměřenou teplotu.
Jako embedded zařízení jsme zvolili STM3210C-Eval board a firmware vytvořili pomocí knihoven
od STM tak, aby byl přenositelný na de-facto libovolné STM32 zařízení.
Vzhledem k náročnosti programování v embedded prostředí jsme zvolili strategii udržet firmware
co nejjednodušší.
Specifikovali jsme několik různých obrazovek, které může STM3210C-Eval board zobrazovat na svém
displeji a díky kterým může uživatel nastavit jednotlivé intervaly, vidět aktuálně naměřenou teplotu a
připojit zařízení k serveru.

\paragraph{Část práce zabývající se webovým serverem}
Vzhledem k povaze embedded zařízení a požadavku na co nejjednodušší firmware, jsme se rozhodli
implementovat webový server jako separátní entitu.
Práci jsme si ulehčili použitím webového backendu Django.
Přestože námi implementovaný webový server nemá příliš rozsáhlou funkcionalitu, mohl by do budoucna
být rozšířen o spoustu dalších vlastností.

\paragraph{Komunikace mezi serverem a embedded zařízením}
V rámci této části práce jsme rozebrali několik možností, jak by šla komunikace mezi webový serverem
a embedded zařízením realizovat.
Nakonec jsme se rozhodli pro implementaci komunikace pouze pomocí HTTP.
Specifikace komunikace by mohla být snadno rozšířena o několik dalších položek, aniž by se měnil její
princip.
Dále jsme rozebrali možnosti jak komunikaci zabezpečit a to jak ve smyslu autentizace STM na serveru,
tak ve smyslu \uv{skrytí} obsahu komunikace pomocí šifrování.
Nakonec jsme ještě rozebrali možnosti synchronizace intervalů mezi serverem a STM.


% shrnutí
\paragraph{Shrnutí}
Přestože v rámci této práce podporujeme \uv{pouze} STM32 zařízení, které serveru posílá pouze naměřenou
teplotu a vyměňuje si s ním nastavení časových intervalů, do specifikace komunikace by se poměrně
snadno daly zařadit i další položky jako je například vlhkost vzduchu.
Dodejme, že pro podporu jiných embedded zařízení by bylo potřeba přeprogramovat hardwarově specifickou
část firmwaru, nicméně na serveru by přidání podpory pro nový typ zařízení nebylo složité.

% použití RTOS a GUI knihovny