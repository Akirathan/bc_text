
% TODO: Přidat sem někam zmínku o rt_assert?

\chapter{Vyhodnocení}
% budeme testovat jenom STM
V následující kapitole rozebereme způsoby, jakým se dá testovat funkcionalita celého systému.
Testování funkcionality serveru není problém, takže se zaměříme pouze na testování STM.

Firmware pro STM se poměrně těžko testuje.
Hlavně kvůli tomu, že na STM není žádný operační systém, který by spouštěl automatické testy třeba jako
samostatné procesy a potom nám vrátil vyhodnocení.


Části firmwaru, které nejsou závislé na hardwaru, je vhodné testovat na PC.
Takovou částí je například celá síťová komunikace - tedy HTTP klient.

% HW nezávislé komponenty
\section{Hardwarově nezávislé komponenty}

% Device simulator
\subsection{Device simulator}
Device simulator je velice jednoduchý program, který simuluje síťovou část STM.
Device simulator původně vznikl jako záměr vytvořit jádro síťové komunikace - tedy HTTP klienta,
parser, a vše co s tím souvisí a to celé poskládat do konzolového programu, který by na vstupu
dostával příkazy typu \uv{připoj se k serveru} nebo \uv{změň teplotu}.
Vybudovat jednoduchý mechanismus zpracování příkazů ze vstupu je snadné, nejtěžší část je síťová
komunikace.
Hlavní \uv{produkt} device simulatoru mělo být jádro síťové komunikace, které bychom mohli přímo použít
na STM.
Mechanismus zpracování příkazů je vedlejším \uv{produktem}, nicméně poměrně užitečným - dovolí nám
to totiž testovat připojení více STM k serveru najednou.
Device simulator umí v podstatě pouze provádět \emph{komunikační cyklus} se serverem.

% výhody devsimu
Výhoda vývoje síťové komunikace v rámci device simulatoru je jasná - můžeme komunikaci testovat
na jednom PC.
Navíc je zde i lepší možnost ladění programu, protože po každé nemusíme program nejdřív nahrávat
do flash paměti STM.

Device simulator je součástí přílohy této práce.

% HW závislé komponenty
\section{Hardwarově závislé komponenty}
Zde se jedná zejména o \emph{Application drivers}, do kterých patří třídy, které s hardware pracují
buď přímo, nebo přes HAL.
Pro tyto komponenty můžeme vytvořit něco jako unit testy, které přilinkujeme do zbytku firmwaru a explicitně
je zavoláme z funkce \texttt{main}.
Je to sice nepraktické, ale v některých případech se to může hodit.
Například tímto způsobem testujeme \texttt{OneWire}.

Takových \uv{unit testů} ale není mnoho.
Jednak kvůli jejich nepraktičnosti, ale zejména kvůli jejich nadbytečnosti.
Tyto \uv{unit testy} by totiž musely testovat zejména \emph{GUI} část firmwaru, která by se tímto způsobem
testovala opravdu těžko.

% Testování celkově funkcionality - tj. testování v reálném prostředí
\section{Testování celkové funkcionality}
Pro otestování celkové funkcionality, tedy STM, serveru a komunikace mezi nimi, je potřeba připravit
prostředí, které se bude podobat reálnému použití.
Server tedy necháme běžet na PC s veřejnou, statickou IP adresou.
STM připojíme do routeru, který je připojen k internetu.
Ovšem nemusí to být vyloženě router, stačí když STM připojíme do jiného PC, na kterém poběží
DHCP server, a bude přeposílat pakety z STM.
V STM je na pevno zakódovaná URL adresa serveru a navíc je na něm implementováno DNS.

Po tomto zapojení už jen stačí manuálně vyzkoušet funkcionalitu serveru i STM.

