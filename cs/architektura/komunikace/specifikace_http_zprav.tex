\subsection{Specifikace HTTP zpráv}

Tato sekce popisuje obsah jednotlivých HTTP zpráv mezi STM a serverem.
Pro jednoduchost zde nejsou téměř vůbec uvedeny obsahy hlaviček.
Důležité jsou zejména URL requestů a těla zpráv, kde konkrétní URL jsou z ilustrativních důvodů
zjednodušeny.
Těla zpráv jsou ještě navíc šifrována ale pro jednoduchost jsou všechny zprávy zobrazeny tak,
jak by vypadaly bez použití jakéhokoli šifrování.

% Connection fáze
\subsubsection{Fáze připojování}
Nejprve je potřeba vyřešit připojení STM k serveru.

\begin{packetstm}
POST /connect
Content-Type: text/plain

<device_id>
\end{packetstm}

\begin{packetserver}
HTTP/1.1 200 OK
Content-Type: text/plain

<server_real_time>
\end{packetserver}

\texttt{server\_real\_time} je timestamp odeslaný serverem sloužící k synchronizaci času
mezi STM a serverem.

% Posílání naměřené teploty
\subsubsection{Odesílání naměřené teploty}
\begin{packetstm}
POST /actual/temp
Content-Type: text/plain

<temp_timestamp>
<temp_float>
\end{packetstm}

Kde \texttt{temp\_timestamp} je timestamp značící kdy byla teplota naměřena a
\texttt{temp\_float} je teplota v desetinné přesnosti.

% Výměna intervalů
\subsubsection{Výměna intervalů}
\begin{packetstm}
GET /config/intervals/timestamp
\end{packetstm}

\begin{packetserver}
HTTP/1.1 200 OK
Content-Type: text/plain

<interval_timestamp>
\end{packetserver}

\texttt{interval\_timestamp} je timestamp intervalů na serveru.
STM porovná přijatý \texttt{interval\_timestamp} s timestampem svých intervalů a pokud má novější nastavení
než server, tak je odešle:
% odesílání intervalů
\begin{packetstm}
POST /config/intervals
Content-Type: application/octet-stream

<intervals_timestamp><intervals>
\end{packetstm}

Na rozdíl od odstatních zpráv je zde \texttt{intervals\_timestamp} v binárních podobě jako 4 bajtové
neznaménkové číslo, formát \texttt{intervals} bude zmíněn později.

Pokud má starší intervaly než server, tak si je od serveru vyžádá:
% přijímání intervalů
\begin{packetstm}
GET /config/intervals
\end{packetstm}

\begin{packetserver}
HTTP/1.1 200 OK
Content-Type: application/octet-stream

<intervals>
\end{packetserver}

\paragraph{Formát intervalů}
Formát intervalů je stejný jako formát, ve kterém si samo STM intervaly ukládá do své EEPROM a
má následující podobu:
\texttt{<from>} \texttt{<to>} \texttt{<temp>}, kde \texttt{from} a \texttt{to} jsou
časy v minutách v rámci dne a \texttt{temp} je teplota.
Každá položka reprezentuje 4 bajtové neznaménkové číslo.
