\subsection{Specifikace HTTP zpráv}

Tato sekce probírá detaily HTTP komunikace mezi STM a webovým serverem.

% Connection fáze
\subsubsection{Fáze připojování}
Nejprve je potřeba vyřešit připojení STM k serveru.
Chceme totiž uživateli dát vědět zda je STM k serveru připojené a uživatel může tím pádem
sledovat aktuálně naměřené hodnoty přes web.

Můžeme poslat požadavek typu:
\begin{packetstm}
POST /controllers/connect
Content-Type: text/plain

<device_id>
\end{packetstm}

\begin{packetserver}
HTTP/1.1 200 OK
[options]

<server_real_time>
\end{packetserver}

\texttt{server\_real\_time} je timestamp odeslaný serverem sloužící k synchronizaci času
mezi STM a serverem (odkaz).

% Posílání naměřené teploty
\subsubsection{Odesílání naměřené teploty}
\begin{packetstm}
POST /actual/temp
Content-Type: text/plain

<temp_timestamp>
<temp_float>
\end{packetstm}

Kde \texttt{temp\_timestamp} je \texttt{timestamp} značící kdy byla teplota naměřena a
\texttt{temp\_float} je teplota v desetinné přesnosti.

% Výměna intervalů
\subsubsection{Výměna intervalů}
\begin{packetstm}
GET /config/intervals/timestamp
\end{packetstm}

\begin{packetserver}
HTTP/1.1 200 OK
[options]

<interval_timestamp>
\end{packetserver}

STM porovná přijatý \texttt{interval\_timestamp} se svým \texttt{timestampem} a pokud má novější nastavení než server,
tak je odešle:
% odesílání intervalů
\begin{packetstm}
POST /config/intervals
Content-Type: application/octet-stream

<intervals_timestamp>
<intervals>
\end{packetstm}

pokud má starší nastavení než server, tak si je od serveru vyžádá:
% přijímání intervalů
\begin{packetstm}
GET /config/intervals
\end{packetstm}

\begin{packetserver}
HTTP/1.1 200 OK
Content-Type: application/octet-stream
[options]

<intervals>
\end{packetserver}

\paragraph{Formát intervalů}
je následující (stejný jako formát, ve kterém si samo STM intervaly ukládá do své EEPROM)
\texttt{<from>} \texttt{<to>} \texttt{<temp>}, kde \texttt{from} a \texttt{to} jsou
časy v sekundách v rámci dne a \texttt{temp} je teplota.
Každá položka reprezentuje 4 bajtové neznaménkové číslo.
