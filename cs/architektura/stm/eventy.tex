
\subsection{Eventy}
V této kapitole rozebereme některé důležité události, které mohou v rámci firmware STM vzniknout.

Mezi tyto události patří:
\begin{itemize}
  % IntervalsChangedStmEvent
  \item Změna intervalů na STM.
    Jako reakci na tuto událost by STM mělo nově nastavené intervaly uložit do EEPROM a dále
    tyto intervaly poslat na server v případě, že na serveru jsou nastavené starší intervaly.

  % IntervalsChangedServerEvent
  \item Příjem intervalů ze serveru o které samo STM požádalo. Reakce na tuto událost je
    uložení intervalů do EEPROM.

  % ConnectEvent
  \item Připojení k serveru. Myšleno ne pouze navázané TCP spojení, ale spojení ve smyslu naší
    aplikace tj. server poslal svůj aktuální čas ve formátu timestampu.
    STM reaguje tak, že si právě přijatý timestamp nastaví jako svůj vlastní čas, tedy
    uloží timestamp do RTC.

  % MeasuredTemperatureEvent
  \item Právě naměřená teplota. Tuto hodnotu chceme zobrazit uživateli na STM a také ji poslat
    na server.

  % CommunicationErrorEvent
  \item Chyba síťové komunikace. Toutu chybou máme na mysli v podstatě libovolnou síťovou chybu
    od vypojení ethernetového kabelu až po nemožnost naparsování HTTP response.
    Na tuto událost STM reaguje restartováním celé komunikace.
    Kvůli složitosti implementace nemá vůbec smysl aby si STM pamatovalo, v jakém stavu chyba
    nastala a zkoušelo komunikace restartovat právě od tohoto momentu.

  % Eth-link UP
  \item Zapojení ethernetového kabelu. Jak jsme již zmiňovali v kapitole analýza, uživateli
    dáváme možnost zapojit nebo odpojit ethernetový kabel za běhu aplikace.
    Zapojení ethernetového kabelu by v ideálním případě mělo ihned odstartovat pokus o spojení se serverem
    a výsledek tohoto pokusu oznámit uživateli.
    Připojit se k serveru může STM pouze pokud má uložený klíč v EEPROM.
    Pro přehlednost a jednoduchost aplikace stanovíme ještě další podmínku, která musí být splněna
    před pokusem o připojení k serveru - aktuální obrazovka musí být \texttt{MainFrame}.

  % KeySavedEvent
  \item Nastavení klíče. Tato událost nastává v momentě, kdy je aktuální obrazovka \texttt{KeyFrame},
    a uživatel stiskne tlačítko \uv{submit}.
    STM na to reaguje uložením klíče do EEPROM a také pokusem o připojení k serveru,
    pokud jsou všechny podmínky pro tento pokus splněny.
\end{itemize}

% Všechny eventy jsou předávané do Application
Každá z výše uvedených událostí je reprezentovaná třídou v programu.
Problém s připojením je například reprezentován třídou \texttt{CommunicationErrorEvent}, nastavení
klíče potom třídou \texttt{KeySavedEvent}, apod.
Události mohou být generované v různých částech programu, \texttt{MeasuredTemperatureEvent} je generovaný
například v \texttt{TempController}.
Všechny tyto události musí být po vzniku předány třídě \texttt{Application}, která v tomto případě
slouží jako centrální autorita ošetřující všechny druhy událostí.

% výhody a nevýhody návrhu
Výhoda tohoto návrh je v tom, že reakce na všechny důležité události máme na jednom místě.
Nevýhoda je potom taková, že nesmíme zapomínat události generovat.
