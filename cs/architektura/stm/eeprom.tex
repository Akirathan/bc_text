
\subsection{EEPROM}
EEPROM používáme jako persistentní datové uložiště, do kterého ukládáme konfiguraci intervalů
a privátní DES klíč.
Kromě samotných intervalů a samotného klíče je ještě potřeba do EEPROM ukládat metadata -
například počet intervalů, timestamp intervalů nebo flag značící jestli klíč je uložen.
Je zbytečně složité kvůli těmto pár datům a metadatům používat nějaký file systém.

% Detaily implementace
Co se týká implementačních detailů, tak \texttt{EEPROM} je třída, která se stará o ukládání
dat do EEPROM a mimo jiné obsahuje definice adres, na které jsou ukládány různá data a metadata.
EEPROM při inicializaci nevyžaduje žádný specifický formát, což mimo jiné znamená, že vymazat
všechna data z EEPROM znamená vyplnit celou EEPROM nulami.

Vymazání celé EEPROM je také něco, co je vhodné udělat předtím než se uživatel STM rozhodne
toto zařízení předat někomu jinému.

