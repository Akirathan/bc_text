\subsection{Tasky}

% Definice tasku
V následujícím textu budeme označovat task jako sémanticky samostatné výpočetní
vlákno.
V podstatě je to totéž jako task v kontextu FreeRTOS.
V STM3210C-EVAL je paralelismus pouze virtuální, protože zabudovaný procesor je
jednojádrový.

% Tasky v naší aplikaci
Naše aplikace potřebuje provádět 3 tasky:
\begin{itemize}
    \item \textbf{GUI task} má na starosti periodicky iterovat přes všechny GUI elementy
        na displeji a v případě potřeby je překreslit.

    \item \textbf{User input task} má na starosti zjišťovat jestli uživatel zadal nějaký
        vstup, ať už je to zmáčknutí joysticku, nebo jiného tlačítka, nebo dotek
        na obrazovku.

        Uživatelský vstup lze zpracovávat i pomocí interrupt handlerů - joystick
        ale i jiné tlačítka lze nastavit tak, aby při stlačení vyvolali interrupt.
        Pro naši situaci to ale není výhodné, protože bychom museli mít více různých
        interrupt handlerů pro každý možný vstup.
        Lepší je nastavit periodický task tak, aby iteroval všechny možné vstupy. 

    \item \textbf{Ethernet task} má na startosti zpracovávání paketů z ethernetové
        periferie. Je potřeba periodicky volat funkci z \textit{LwIP} knihovny [LwIP doc].
        Zpracovávat ethernetové pakety z interrupt handleru není doporučeno [LwIP doc].

    \item \textbf{RTC task} má na starosti zpracovávání sekundového interruptu a jednou za
        minutu změřit aktuální teplotu. RTC task nemá s RTOS taskem nic společného, je
        to totiž interrupt handler.
\end{itemize}

% RTOS vs timery
% ...