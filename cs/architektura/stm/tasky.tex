\subsection{Tasky}

% Definice tasku
V následujícím textu budeme označovat \emph{task} jako sémanticky samostatné výpočetní
vlákno.
V podstatě je to totéž jako task v kontextu RTOS.
V STM je paralelismus pouze virtuální, protože procesor je jednojádrový.

% Tasky v naší aplikaci
Naše aplikace potřebuje provádět 3 tasky:
\begin{itemize}
  \item \textbf{GUI task} má na starosti periodicky iterovat přes všechny GUI elementy
    na displeji a v případě potřeby je překreslit.

  \item \textbf{User input task} má na starosti zjišťovat jestli uživatel zadal nějaký
    vstup, ať už je to zmáčknutí joysticku, nebo jiného tlačítka.

    Uživatelský vstup lze zpracovávat i pomocí interrupt handlerů - joystick
    ale i jiná tlačítka lze nastavit tak, aby při stlačení vyvolali interrupt.
    Pro naši situaci to ale není výhodné, protože bychom museli mít více různých
    interrupt handlerů pro každý možný vstup.
    Lepší je nastavit periodický task tak, aby iteroval všechny možné vstupy.

  \item \textbf{Ethernet task} má na startosti zpracovávání paketů z ethernetové
    periferie. Je potřeba periodicky volat dvě funkce z \textit{LwIP} knihovny -
    \texttt{ethernetif\_input}, která se stará o nízko-úrovňový příjem nebo odesílání paketů resp.
    bytů z ethernetové periferi, a \texttt{sys\_check\_timeouts}, která kontroluje platnost různých
    dočasných hodnot, jako je třeba ARP tabulka, a v případě potřeby je obnoví.
    Proto je vhodné tento task provádět z mainloop.

  \item \textbf{RTC task} má na starosti zpracovávání sekundového interruptu a jednou za
    minutu změřit aktuální teplotu.
\end{itemize}

% Definice mainloop
Vzhledem k tomu, že firmware je jediné, co v STM poběží, je potřeba aby svůj běh neukončil
hned po prvním vykreslení obrazovky resp. zpracování pár paketů.
Potřebovali bychom mít někde ve firmware nekonečný cyklus, který bude zpracovávat naše
tasky.
Takovému nekonečnému cyklu budeme říkat \emph{mainloop}.
V rámci firmware pro embedded zařízení, který nemá RTOS, je použití \emph{mainloop} běžné.

% definice interrupt handler kontextu
Kromě \emph{mainloop} můžou být tasky vykonávané ještě v rámci interrupt handleru některé
periferie, například časovače, nebo stisknutí tlačítka.
Interrupt handlery mají obecně tu výhodu, že jsou spuštěny vždy po určité události -
v případě časovače je to uplynutí periody.

% mainloop vs interrupt kontext
Ne všechny tasky je vhodné provádět v kontextu interrupt handleru.
Například podle dokumentace LwIP není doporučeno volat jakoukoli funkci, která má odesílat
nebo přijímat pakety z interrupt handleru.
Proto je potřeba ethernet task provádět vždy v rámci mainloop.
% User input task je vhodný provádět z interrupt handleru
Naopak user input task je vhodné provádět v rámci interrupt handleru časovače nastaveného
na rozumnou periodu.
Pokud by se totiž měl provádat z mainloop, není jasné jak přesně by měl fungovat, protože čtení
vstupu od uživatele je samo o sobě blokující funkce.


