
\section{Práce se systémem}
V této kapitole ukážeme, jak se systém používá jako celek.
Konkrétně ukážeme a popíšeme fotografii STM se zapojeným teplotním sensorem a relé modulem a dále
screenshot webových stránek.

% STM
\subsection{STM}

% TODO: fotka STM s teplotním senzorem a relé modulem, v MainFrame

Na obrázku (TODO: ref) vidíme dvě STM zařízení.
To s displejem je STM3210C-Eval board, které používáme v celé práci.
% discovery jako debugger
To druhé je STM32F4-Discovery, které používáme jen kvůli tomu, že je na něm zabudovaný ST-Link
debugger (TODO: ref), přes který jednak nahráváme firmware do STM3210C-Eval boardu a jednak
ho používáme jako debugger.
V reálném použití nebude použití STM32F4-Discovery nutné.

% teplotní sensor, relé modul, joystick, reset button, ...
Do STM3210C-Eval boardu (dále jen STM) je zapojen teplotní senzor Dallas DS18B20 (TODO: ref)
a relé modul (TODO: ref).
Konkrétní zapojení do GPIO pinů se dá velice snadno změnit ve zdrojovém kódu.
Dole pod displejem STM jsou zleva vidět čtyři LED diody, černé RESET tlačítko, joystick ...
Z toho reálně využíváme pouze joystick jako vstup a červenou diodu jako notifikaci o tom, že
se v STM stala nějaká chyba.
RESET tlačítko slouží jako restart firmware STM.

Na displeji je aktuálně zobrazen \texttt{MainFrame}.


% Web server
\subsection{Webový server}

\subsubsection{Zobrazení všech STM}

% TODO: screenshot devices stránky (stačí když tam bude jen jedno zařízení). Temprature má hvězdičku.

Na stránce (TODO: ref) je vidět jedno zařízení s ID "stm1", které je zrovna online, s rozbalenými
detaily.
U položky temperature je červená hvězdička, která značí že server přijal od STM novou hodnotu, která
je připravena k aktualizaci pomocí "refresh" tlačítka.
U položky intervals je kromě refresh tlačítka vidět i edit tlačítko, pomocí kterého můžeme intervaly
libovolně editovat, později uložit a poslat do STM.
Uložení intervalů a jejich následné poslání do STM na screenshotu sice vidět není, nicméně funguje to
tak, že po editaci a uložení intervalů se zobrazí tlačítko "save into device", pomocí kterého můžeme
intervaly synchronizovat s STM.
U položky intervals je také vidět timestamp.

Dodejme ještě, že takto zobrazených zařízení může na stránce být víc.

% Přidání nového zařízení
\subsubsection{Přidání nového zařízení}
Přidání nového zařízení je proces, při kterém server vygeneruje nový privátní DES klíč, který uživatel
zadá do STM a to dále šifruje zprávy pomocí tohoto klíče.

Detailněji je tento proces specifikován v kapitole (TODO: ref analýza/komunikace).

Na webový stránkách stačí kliknout na tlačítko "register new device" a později na "generate new key".


