\chapter{Analýza}

V této kapitole rozebereme možnosti návrhu celého systému.
...

%%%%%%%%%%%%%%%%%%%%%%%%%%%%%%%%%%%%%%%%%%%%%%%%%%%%%%%%%%%%%%%%%%%%%%%%%%%%%%%%
%%%%%%%%%%%%%%%%%%%%%%%%%%  web server %%%%%%%%%%%%%%%%%%%%%%%%%%%%%%%%%%%%%%%%%%
%%%%%%%%%%%%%%%%%%%%%%%%%%%%%%%%%%%%%%%%%%%%%%%%%%%%%%%%%%%%%%%%%%%%%%%%%%%%%%%%%
\section{Web server}

\subsection{Funkcionalita}
% Obecné věci
Webový server slouží jako dálkové rozhraní mezi uživatelem a zařízením.
Uživatel může odkudkoli kde má připojení k internetu zkontrolovat činnost zařízení
a případně ho přenastavit.

Server podporuje pouze jeden druh zařízení (STM smart heating), nicméně je naprogramován
tak, že pro případné rozšíření na více druhů zařízení nemusíme v programu dělat veliké
změny.

\subsection{Použité technologie}

\subsubsection{Backend}
% Backend framework a proč ho používat?
Přestože očekáváme, že koečná webová aplikace bude poměrně malá a to jak požadovanou funkcionalitou, tak svým
vzhledem, použitím některého z webových backend frameworků nám může velice usnadnit práci, například v těchto
aspektech:
\begin{itemize}
    \item Zabudovaná podpora uživatelů - nám jen stačí přidat uživatelům odkazy na jejich
        zařízení.
    \item Template system nám umožňuje v HTML stránkách používat speciální
        tagy, díky kterým můžeme například do HTML stránky vložit hodnotu proměnné, nebo ... (TODO)
        Template system nakonec vygeneruje validní HTML stránku.
    \item Zabudovaná podpora databázových systémů - většina backend frameworků poskytuje vrstvu abstrakce nad
        databázovýmy(?) systémy. V praxi to znamená, že nikde v kódu nemusíme přímo vytvářet SQL dotazy,
        stačí používat funkční API pro databáze v programovacím jazyce, ve kterém je daný framework napsaný.
\end{itemize}

% Různé frameworky a jaký je mezi nimi rozdíl
Existuje mnoho různých backend frameworků.
Jsou to například NodeJS napsaný v Javascriptu, Ruby on Rails napsaný v Ruby, Django napsaný v Pythonu a
Laravel napsaný v PHP. % TODO: formátovat jednotlivé názvy, případně přidat reference
Všechny tyté frameworky nám mohou usnadnit práci podobným způsobem, pro naši aplikaci se liší především
tím jaký programovací jazyk používají.

% Výběr Djanga
Vzhledem k tomu, že pro naši aplikaci je pro výběr konkretního frameworku relevantní pouze to jaký programovací
jazyk používá, a vzhledem k tomu, že autor má předchozí zkušenosti s Pythonem, vybereme Django.

\subsubsection{Frontend}
% Bootstrap
Vytváření vzhledu webu si usnadníme použitím knihovny Bootstrap.
Ta nám umožňuje snadno vytvořit moderně vypadající responsive webovou aplikaci. % TODO: co znamená responsive?

% Device stránka bude interaktivní, protože na ní budou všechny zařízení
...

\section{Komunikace STM a web serveru}

% Jakou funkcionalitu bude náš systém potřebovat?
Náš komunikační systém bude potřebovat následující funkcionalitu:
\begin{itemize}
    \item Periodicky posílat aktuálně naměřenou teplotu z STM na server.
    \item Synchronizovat čas mezi serverem a STM
    \item Synchronizovat nastavení intervalů na STM a na serveru.
          Pro tento bod potřebujeme mít funkční synchronizaci času.
\end{itemize}

% definice actual a config
Dále budeme výše zmíněné měřitelné hodnoty, které STM měří a periodicky je posílá
serveru, nazývat [actual] a hodnoty, které se dají nastavit jak na serveru, tak na
STM [config] hodnoty.
% předpoklad přidávání nových actual a config položek
Při návrhu komunikace předpokládáme, že budeme chtít přidávat další config a actual
položky, takže v následujícím textu jsou intervaly zaměnitelné s jakoukoli
config položkou, totéž platí pro naměřenou teplotu a actual položky.

% STM a dynamická IP adresa
Předpokládáme, že webový server běží na statické, veřejné IP adrese.
Pokud STM připojíme ethernetovým kabelem do internetové sítě, bude mu přidělena dynamická IP adresa.
Nemá smysl aby uživatel, kromě toho že koupí STM, pořizoval i statickou IP adresu.
Vzhledem k tomu, že IP adresa STM je dynamická, server ji nemůže dopředu znát.
Což znamená, že iniciátor komunikace musí být STM.

% Volba HTTP vs custom protocol
Pro komunikaci mezi webovým serverem a STM (tedy koncovým zařízením) můžeme vybrat buď HTTP nebo vlastní protokol.
Implementace a otestování vlastního protokolu by bylo příliš pracné, využijeme tedy existujícího HTTP.
Navíc nám v takovém případě stačí implementovat HTTP pouze na straně STM.

%%%%%%%%%%%%%%%%%%%%%%%%%%%%%%%%%%%%%%%%%%%%%%%%%%%%%%%%%%%%%%%%%%%%%%%%%%%%%%%%%%%%%%%%%%%%
%                              SYNCHRONIZACE                                               %
%%%%%%%%%%%%%%%%%%%%%%%%%%%%%%%%%%%%%%%%%%%%%%%%%%%%%%%%%%%%%%%%%%%%%%%%%%%%%%%%%%%%%%%%%%%%

% Synchronizace config položek
Config data může uživatel nastavit jak na serveru, tak přímo na STM.
Proto je potřeba zavést synchronizaci.

\subsection{Synchronizace}

% nastavovat se můžu naráz na serveru i na STM
Uživateli dovolíme nastavovat config data naráz na serveru i na STM, platit bude ale pouze
nastavení, které bylo uloženo později (myšleno vzhledem k reálnému času).
Jiná, uživatelsky pravděpodobně příjemnější, možnost by byla nedovolit uživateli současně
nastavovat stejná config data na serveru i na STM.
Tato možnost je ovšem zbytečně složitá na implementaci.

% Synchronizace času - možnosti a proč je potřeba?
Pro synchronizaci config dat je potřeba, aby server i STM měli nastavený společný čas.
Vzhledem k tomu, že implementace \emph{NTP} (Network time protocol) ale i jednoduššího a podobného
protokolu \emph{SNTP} (Simple Network Time Protocol) je na STM pracná, zvolíme jiný, jednodušší způsob synchronizace
času mezi STM a serverem, který bude vycházet z \emph{TP} (Time protocol).

\subsubsection{Synchronizace času}
% vyjdeme z TP a upravíme ho pro naše potřeby
V rámci synchronizace času postačí sekundové rozlišení.
Vyjdeme z \emph{TP}, který má sekundové rozlišení, a ještě ho zjednodušíme pro naše účely.
Podobně jako je tomu u \emph{TP}, budeme ze serveru posílat počet sekund, které
uběhly od nějakého pevně stanoveného data.
Na rozdíl od \emph{TP} to ale ještě zjednodušíme tak, že tento počet sekund bude posílat server
v rámci první HTTP odpovědi.
Další rozdíl od \emph{TP} bute ten, že naše počáteční datum nebude \texttt{1.1.1900}, ale \texttt{1.1.1970}.
% definice timestamp
\textbf{Timestampem} budeme dále uvažovat číslo reprezentující počet sekund od \texttt{1.1.1970 00:00},
které se vejde do 4-bytové proměnné bez znamenénka.

% uživatel musí nastavit čas při zapnutí
Při zapnutí STM musí uživatel nastavit čas. (TODO: viz ...)
Nemá možnost nastavit datum - to je implicitně nastaveno na \texttt{1.1.1970} (dále označováno jako \texttt{REF\_DATE}).
Je to nutné kvůli tomu, aby STM vědělo který interval je zrovna nastavený.
Uživatel může nastavit libovolný čas, který se nijak neshoduje s realitou, jde jen o to,
aby STM vědělo, jakou teplotu má udržovat.
% correct_bit
Když uživatel nastaví config data (intervaly), tak k nim STM uloží aktuální timestamp
a ještě k tomu příznak, který značí, že timestamp je nesynchronizovaný.
Tento příznak budeme dále označovat jako \textbf{correct\_bit}.

% Problém zpožděného času po synchronizaci
V souvislosti se vzdáleností STM od serveru a se zpožděním síťové komunikace se může
stát, že čas na STM bude o pár sekund opožděný (nemělo by to ale být více než 2 sekundy -
viz \texttt{[packet\_over\_world\_benchmark]}).
Kdyby se uživateli například podařilo na serveru uložit nové nastavení intervalů
a pár mikro sekund poté nastavit nové intervaly na STM, mělo by STM správně nahrát
intervaly na server, namísto toho ale intervaly stáhne ze serveru a svoje nastavení přepíše.
Předpokládejme, že uživatel toto dělat nebude.



\subsection{Zabezpečení}

% Co požadujeme
\paragraph{Požadavky}
\begin{itemize}
    \item Autentikace zpráv: potřebujeme zajistit, aby server přijímal pouze zprávy od STM a nikoho jiného.
    \item Šifrování zpráv: šifrovat budeme jak hlavičky, tak těla HTTP zpráv. Z hlediska bezpečnosti
        je to lepší než šifrovat pouze těla.
\end{itemize}


% použití HTTPS?
\paragraph{HTTPS}
Komunikace webového serveru s STM by mohla být kompletně vedena v HTTPS.
Pro HTTPS potřebujeme na STM TLS.
To je poskytováno například knihovnou Cyclone-SSL, která je závislá na knihovně Cyclone-TCP/IP.
My už jsme ovšem pro STM použili knihovnu LwIP, protože STM poskytuje ethernetový driver přímo
pro tuto knihovnu a tím pádem můžeme LwIP rovnou použít.
Kdybychom ale chtěli použít Cyclone-TCP/IP, tak bychom museli ještě implementovat driver pro
tuto knihovnu.
Závěr je tedy takový, že HTTPS používat nebudeme.

% Symetrické VS asymetrické šifrování
\paragraph{Symetrické nebo asymetrické šifrování}
Kdybychom chtěli použít pouze \emph{asymetrické} šifrování, server by měl privátní klíč a každé STM
by šifrovalo komunikaci veřejným klíčem.
Problém tohoto řešení je nedostatečná autentizace - veřejný klíč může získat každý, nejen STM.
Pro naše účely je tedy lepší použít \emph{symetrické} šifrování.
% jaké šifrování? --> DES
Potřebujeme, aby uživatel mohl pohodlně zadat klíč do STM.
Bude se nám tedy hodit šifrování s krátkým klíčem, což je například DES, které má
klíč délky 8 bytů.
Pro naše využití to je dostačující zabezpečení.


% Kde vzít klíče?
Pro symetrické šifrování potřebujeme, aby existovalo tolik párů klíčů, jako je STM zařízení.
Jeden z páru má server a druhý STM.
Následující text se zabývá problémem kde a kdy vzít tyto klíče.
\\ \\ 
Máme dvě možnosti:
\begin{itemize}
    \item STM má hardcoded klíč ve svém FW a server má databází všech těchto klíčů.
    \item Server generuje klíče na žádost uživatele.
\end{itemize}

% Proč je hardcoded klíč špatný nápad?
Mít klíč zabudovaný do FW STM není příliš vhodné pro případy, kdy by si uživatel pořídil STM
"z druhé ruky".
Zejména je zde nižší zabezpečení nového vlastníka STM, bývalý vlastník by totiž teoreticky
mohl z FW dostat podobu klíče a podvrhovat komunikaci.

Proto zvolíme řešení s generováním nových klíčů na žádost uživatele.

% Generování klíčů
\subsubsection{Generování klíčů}

% Obecné vlastnosti generování klíčů
Nevýhoda generování klíčů na serveru je ta, že server nemůže vědět, kterému STM má klíč přiřadit.

% Jak server může přiřadit klíče jednotlivým STM?
% predem, resp. až po přijetí connect
Server si ke každému STM potřebuje přiřadit klíč.
I kdyby dopředu věděl, který klíč patří kterému STM, nemůže vědět od kterého STM mu přišla
\emph{connect} zpráva (což je první zpráva, kterou STM pošle - viz ...), protože dopředu nezná
IP adresy jednotlivých STM.
Což znamená, že není důležité, jestli má klíče k zařízením přiřazené předem, nebo jestli
si je přiřadí až po tom, co od jednotlivých zařízení dostane \emph{connect} zprávu.

% Postup generování klíčů
\subparagraph{Postup generování klíčů}
\begin{enumerate}
    \item Server vygeneruje klíč s omezenou životností na žádost uživatele.
    \item Uživatel zadá klíč do STM.
    \item STM pošle šifrovanou \emph{connect} zprávu serveru.
    \item Server bude dešifrovat zprávu všemi dosud nespárovanými klíči, dokud nedostane korektní
        \emph{connect} zprávu tj. korektní HTTP hlavičku s tělem obsahujícím existující Device ID.
    \item Server má sprárované Device ID s klíčem a IP adresou.
\end{enumerate}


% UID (TODO: tohle bude v architektuře)
%Každé STM zařízení má v sobě uložený 96-bitový unikátní identifikátor.
